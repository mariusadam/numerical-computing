\documentclass{article}%
\usepackage{amsmath}
\usepackage{amsfonts}
\usepackage{amssymb}
\usepackage{graphicx}%
\setcounter{MaxMatrixCols}{30}
%TCIDATA{OutputFilter=latex2.dll}
%TCIDATA{Version=5.50.0.2953}
%TCIDATA{CSTFile=40 LaTeX article.cst}
%TCIDATA{Created=Tuesday, May 20, 2014 11:40:15}
%TCIDATA{LastRevised=Tuesday, May 20, 2014 13:59:01}
%TCIDATA{<META NAME="GraphicsSave" CONTENT="32">}
%TCIDATA{<META NAME="SaveForMode" CONTENT="1">}
%TCIDATA{BibliographyScheme=Manual}
%TCIDATA{<META NAME="DocumentShell" CONTENT="Standard LaTeX\Blank - Standard LaTeX Article">}
%BeginMSIPreambleData
\providecommand{\U}[1]{\protect\rule{.1in}{.1in}}
%EndMSIPreambleData
\newtheorem{theorem}{Theorem}
\newtheorem{acknowledgement}[theorem]{Acknowledgement}
\newtheorem{algorithm}[theorem]{Algorithm}
\newtheorem{axiom}[theorem]{Axiom}
\newtheorem{case}[theorem]{Case}
\newtheorem{claim}[theorem]{Claim}
\newtheorem{conclusion}[theorem]{Conclusion}
\newtheorem{condition}[theorem]{Condition}
\newtheorem{conjecture}[theorem]{Conjecture}
\newtheorem{corollary}[theorem]{Corollary}
\newtheorem{criterion}[theorem]{Criterion}
\newtheorem{definition}[theorem]{Definition}
\newtheorem{example}[theorem]{Example}
\newtheorem{exercise}[theorem]{Exercise}
\newtheorem{lemma}[theorem]{Lemma}
\newtheorem{notation}[theorem]{Notation}
\newtheorem{problem}[theorem]{Problema}
\newtheorem{proposition}[theorem]{Proposition}
\newtheorem{remark}[theorem]{Remark}
\newtheorem{solution}[theorem]{Solution}
\newtheorem{summary}[theorem]{Summary}
\newenvironment{proof}[1][Proof]{\noindent\textbf{#1.} }{\ \rule{0.5em}{0.5em}}
\begin{document}
\section{Subiectul 1}
 
\begin{problem}
Se consider\u{a} sistemul%
\[
\left[
\begin{array}
[c]{cc}%
2 & 1\\
-1 & 2
\end{array}
\right]  x=\left[
\begin{array}
[c]{c}%
3\\
1
\end{array}
\right]  .
\]


\begin{enumerate}
\item S\u{a} se studieze convergen\c{t}a metodei lui Jacobi.

\item De c\^{a}te itera\c{t}ii este nevoie pentru a aproxima solu\c{t}ia cu
eroarea absolut\u{a} $\varepsilon$ dat\u{a}, dac\u{a} se ia $x^{(0)}%
=[0,0]^{T}.$
\end{enumerate}
\end{problem}

\begin{problem}
S\u{a} se determine o formul\u{a} de cuadratur\u{a} de forma
\[
\int_{-1}^{1}\sqrt{1-x^{2}}f(x)\mathrm{d}\,x=A_{1}f(1)+A_{2}f(x_{1})+R(f)
\]
care s\u{a} aib\u{a} grad maxim de exactitate.
\end{problem}

\begin{problem}
\begin{enumerate}
\item Se consider\u{a} ecua\c{t}ia \^{\i}n $\mathbb{R}$ $f(x)=0$ cu
r\u{a}d\u{a}cina $\alpha$ \c{s}i o metod\u{a} cu ordinul de
convergen\c{t}\u{a} $p$ \c{s}i eroarea asimptotic\u{a} $C_{p}$. Dac\u{a} se
fac $N_{p}$ opera\c{t}ii pe pas \c{s}i opera\c{t}iile de ini\c{t}ializare se
ignor\u{a}, atunci num\u{a}rul total de opera\c{t}ii necesar pentru a aproxima
solu\c{t}ia cu precizia $\varepsilon$ este%
\begin{equation}
T_{p}=\frac{N_{p}}{\log p}\log\left[  \frac{\frac{\log C_{p}}{p-1}%
+\log\varepsilon}{\frac{\log C_{p}}{p-1}+\log e_{0}}\right]  ,\tag{**}%
\end{equation}
unde baza logaritmului este arbitrar\u{a}, $e_{0}$ este eroarea ini\c{t}ial\u{a}.

\item Se consider\u{a} calculul lui $\sqrt{a}$, $a>0$, cu metoda lui Newton, astfel:

\begin{enumerate}
\item Se pune $a=2^{2m}r$, $1/4\leq r<1$.

\item Se calculeaz\u{a} $\sqrt{r}$ cu recuren\c{t}a
\[
x_{n+1}=\frac{1}{2}\left(  x_{n}+\frac{r}{x_{n}}\right)
\]
p\^{a}n\u{a} se atinge precizia $\varepsilon.$

\item Rezultatul returnat este $z=2^{m}x_{N}$, unde $x_{N}$ este rezultatul de
la punctul b.
\end{enumerate}
\end{enumerate}

Da\c{t}i o margine superioar\u{a} a num\u{a}rului de itera\c{t}ii \c{s}i de
opera\c{t}ii de la punctul (2.b) pentru a ob\c{t}ine precizia $eps$
(epsilon-ul ma\c{s}inii) folosind formula (**).

\emph{Indica\c{t}ie}: Fie $e_{n}=|x_{n}-\alpha|$ eroarea la pasul $n$. Se pune
$e_{n+1}\approx C_{p}e_{n}^{p}$. Din condi\c{t}ia $e_{n}\approx\varepsilon$
$\ $se scoate $n$. Eroarea asimptotic\u{a} pentru metoda lui Newton este
$\frac{f^{\prime\prime}(\alpha)}{2f^{\prime}(\alpha)}$.
\end{problem}

\newpage

\section{Subiectul 2}



\begin{problem}
Se consider\u{a} sistemul%
\[
\left[
\begin{array}
[c]{cc}%
2 & 1\\
1 & 2
\end{array}
\right]  x=\left[
\begin{array}
[c]{c}%
3\\
1
\end{array}
\right]  .
\]


\begin{enumerate}
\item S\u{a} se studieze convergen\c{t}a metodei lui Jacobi.

\item De c\^{a}te itera\c{t}ii este nevoie pentru a aproxima solu\c{t}ia cu
eroarea absolut\u{a} $\varepsilon$ dat\u{a}, dac\u{a} se ia $x^{(0)}%
=[0,0]^{T}.$
\end{enumerate}
\end{problem}

\begin{problem}
S\u{a} se determine o formul\u{a} de cuadratur\u{a} de forma
\[
\int_{-1}^{1}\frac{1}{\sqrt{1-x^{2}}}f(x)\mathrm{d}\,x=A_{1}f(1)+A_{2}%
f(x_{1})+R(f)
\]
care s\u{a} aib\u{a} grad maxim de exactitate.
\end{problem}

\begin{problem}
\begin{enumerate}
\item Se consider\u{a} ecua\c{t}ia \^{\i}n $\mathbb{R}$ $f(x)=0$ cu
r\u{a}d\u{a}cina $\alpha$ \c{s}i o metod\u{a} cu ordinul de
convergen\c{t}\u{a} $p$ \c{s}i eroarea asimptotic\u{a} $C_{p}$. Dac\u{a} se
fac $N_{p}$ opera\c{t}ii pe pas \c{s}i opera\c{t}iile de ini\c{t}ializare se
ignor\u{a}, atunci num\u{a}rul total de opera\c{t}ii necesar pentru a aproxima
solu\c{t}ia cu precizia $\varepsilon$ este%
\begin{equation}
T_{p}=\frac{N_{p}}{\log p}\log\left[  \frac{\frac{\log C_{p}}{p-1}%
+\log\varepsilon}{\frac{\log C_{p}}{p-1}+\log e_{0}}\right]  ,\tag{**}%
\end{equation}
unde baza logaritmului este arbitrar\u{a}, $e_{0}$ este eroarea ini\c{t}ial\u{a}.

\item Se consider\u{a} calculul lui $\sqrt[3]{a}$, $a>0$, cu metoda lui
Newton, astfel:

\begin{enumerate}
\item Se pune $a=2^{3m}r$, $1/8\leq r<1$.

\item Se calculeaz\u{a} $\sqrt{r}$ cu recuren\c{t}a
\[
x_{n+1}=\frac{2}{3}\left(  x_{n}+\frac{r}{2x_{n}^{2}}\right)
\]
p\^{a}n\u{a} se atinge precizia $\varepsilon.$

\item Rezultatul returnat este $z=2^{m}r_{N}$, unde $r_{N}$ este rezultatul de
la punctul b.
\end{enumerate}
\end{enumerate}

Da\c{t}i o margine superioar\u{a} a num\u{a}rului de itera\c{t}ii \c{s}i de
opera\c{t}ii de la punctul (2.b) pentru a ob\c{t}ine precizia $eps$
(epsilon-ul ma\c{s}inii) folosind formula (**).

\emph{Indica\c{t}ie}: Fie $e_{n}=|x_{n}-\alpha|$ eroarea la pasul $n$. Se pune
$e_{n+1}\approx C_{p}e_{n}^{p}$. Din condi\c{t}ia $e_{n}\approx\varepsilon$
$\ $se scoate $n$. Eroarea asimptotic\u{a} pentru metoda lui Newton este
$\frac{f^{\prime\prime}(\alpha)}{2f^{\prime}(\alpha)}$.
\end{problem}

\newpage

\section{Subiectul 3}



\begin{problem}
Se consider\u{a} sistemul%
\[
\left[
\begin{array}
[c]{cc}%
2 & 1\\
1 & 2
\end{array}
\right]  x=\left[
\begin{array}
[c]{c}%
3\\
1
\end{array}
\right]  .
\]


\begin{enumerate}
\item S\u{a} se studieze convergen\c{t}a metodei Gauss-Seidel.

\item De c\^{a}te itera\c{t}ii este nevoie pentru a aproxima solu\c{t}ia cu
eroarea absolut\u{a} $\varepsilon$ dat\u{a}, dac\u{a} se ia $x^{(0)}%
=[0,0]^{T}.$
\end{enumerate}
\end{problem}

\begin{problem}
S\u{a} se determine o formul\u{a} de cuadratur\u{a} de forma
\[
\int_{-1}^{1}\frac{1}{\sqrt{1-x^{2}}}f(x)\mathrm{d}\,x=A_{1}f(-1)+A_{2}%
f(x_{1})+R(f)
\]
care s\u{a} aib\u{a} grad maxim de exactitate.
\end{problem}

\begin{problem}
\begin{enumerate}
\item Se consider\u{a} ecua\c{t}ia \^{\i}n $\mathbb{R}$ $f(x)=0$ cu
r\u{a}d\u{a}cina $\alpha$ \c{s}i o metod\u{a} cu ordinul de
convergen\c{t}\u{a} $p$ \c{s}i eroarea asimptotic\u{a} $C_{p}$. Dac\u{a} se
fac $N_{p}$ opera\c{t}ii pe pas \c{s}i opera\c{t}iile de ini\c{t}ializare se
ignor\u{a}, atunci num\u{a}rul total de opera\c{t}ii necesar pentru a aproxima
solu\c{t}ia cu precizia $\varepsilon$ este%
\begin{equation}
T_{p}=\frac{N_{p}}{\log p}\log\left[  \frac{\frac{\log C_{p}}{p-1}%
+\log\varepsilon}{\frac{\log C_{p}}{p-1}+\log e_{0}}\right]  ,\tag{**}%
\end{equation}
unde baza logaritmului este arbitrar\u{a}, $e_{0}$ este eroarea ini\c{t}ial\u{a}.

\item Se consider\u{a} calculul lui $\frac{1}{a}$, $a>0$, cu metoda lui
Newton, astfel:

\begin{enumerate}
\item Se pune $a=2^{m}r$, $1/2\leq r<1$.

\item Se calculeaz\u{a} $1/r$ cu recuren\c{t}a
\[
x_{n+1}=x_{n}\left(  2-rx_{n}\right)
\]
p\^{a}n\u{a} se atinge precizia $\varepsilon.$

\item Rezultatul returnat este $z=2^{-m}r_{N}$, unde $r_{N}$ este rezultatul
de la punctul b.
\end{enumerate}
\end{enumerate}

Da\c{t}i o margine superioar\u{a} a num\u{a}rului de itera\c{t}ii \c{s}i de
opera\c{t}ii de la punctul (2.b) pentru a ob\c{t}ine precizia $eps$
(epsilon-ul ma\c{s}inii) folosind formula (**).

\emph{Indica\c{t}ie}: Fie $e_{n}=|x_{n}-\alpha|$ eroarea la pasul $n$. Se pune
$e_{n+1}\approx C_{p}e_{n}^{p}$. Din condi\c{t}ia $e_{n}\approx\varepsilon$
$\ $se scoate $n$. Eroarea asimptotic\u{a} pentru metoda lui Newton este
$\frac{f^{\prime\prime}(\alpha)}{2f^{\prime}(\alpha)}$.
\end{problem}

\newpage

\section{Subiectul 4}



\begin{problem}
Se consider\u{a} sistemul%
\[
\left[
\begin{array}
[c]{cc}%
2 & 1\\
-1 & 2
\end{array}
\right]  x=\left[
\begin{array}
[c]{c}%
3\\
1
\end{array}
\right]  .
\]


\begin{enumerate}
\item S\u{a} se studieze convergen\c{t}a metodei Gauss-Seidel.

\item De c\^{a}te itera\c{t}ii este nevoie pentru a aproxima solu\c{t}ia cu
eroarea absolut\u{a} $\varepsilon$ dat\u{a}, dac\u{a} se ia $x^{(0)}%
=[0,0]^{T}.$
\end{enumerate}
\end{problem}

\begin{problem}
S\u{a} se determine o formul\u{a} de cuadratur\u{a} de forma
\[
\int_{-1}^{1}\frac{1}{\sqrt{1-x^{2}}}f(x)\mathrm{d}\,x=A_{1}f(-1)+A_{2}%
f(x_{1})+R(f)
\]
care s\u{a} aib\u{a} grad maxim de exactitate.
\end{problem}



\begin{problem}
\begin{enumerate}
\item Se consider\u{a} ecua\c{t}ia \^{\i}n $\mathbb{R}$ $f(x)=0$ cu
r\u{a}d\u{a}cina $\alpha$ \c{s}i o metod\u{a} cu ordinul de
convergen\c{t}\u{a} $p$ \c{s}i eroarea asimptotic\u{a} $C_{p}$. Dac\u{a} se
fac $N_{p}$ opera\c{t}ii pe pas \c{s}i opera\c{t}iile de ini\c{t}ializare se
ignor\u{a}, atunci num\u{a}rul total de opera\c{t}ii necesar apentru a
aproxima solu\c{t}ia cu precizia $\varepsilon$ este%
\begin{equation}
T_{p}=\frac{N_{p}}{\log p}\log\left[  \frac{\frac{\log C_{p}}{p-1}%
+\log\varepsilon}{\frac{\log C_{p}}{p-1}+\log e_{0}}\right]  , \tag{**}%
\end{equation}
unde baza logaritmului este arbitrar\u{a}, $e_{0}$ este eroarea ini\c{t}ial\u{a}.

\item Compara\c{t}i urm\u{a}toarele dou\u{a} metode pentru calculul \ lui
$\sqrt{r}$ (num\u{a}rul de opera\c{t}ii) folosind formula (**). C\^{a}nd este
de preferat una alteia?%
\begin{align*}
x_{n+1}  & =\frac{1}{2}\left(  x_{n}+\frac{r}{x_{n}}\right)  ,~p=2\\
y_{n+1}  & =\frac{y_{n}\left(  y_{n}^{2}+r\right)  }{3y_{n}^{2}+r},~p=3
\end{align*}

\end{enumerate}

\emph{Indica\c{t}ie}: Fie $e_{n}=|x_{n}-\alpha|$ eroarea la pasul $n$. Se pune
$e_{n+1}\approx C_{p}e_{n}^{p}$. Din condi\c{t}ia $e_{n}\approx\varepsilon$
$\ $se scoate $n$. Eroarea asimptotic\u{a} pentru metoda lui Newton este
$\frac{f^{\prime\prime}(\alpha)}{2f^{\prime}(\alpha)}$.
\end{problem}


\end{document}