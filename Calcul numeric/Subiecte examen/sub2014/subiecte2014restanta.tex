\documentclass[landscape,twocolumn]{article}%
\usepackage{amsmath}
\usepackage{amsfonts}
\usepackage{amssymb}
\usepackage{graphicx}%
\setcounter{MaxMatrixCols}{30}
%TCIDATA{OutputFilter=latex2.dll}
%TCIDATA{Version=5.50.0.2953}
%TCIDATA{CSTFile=40 LaTeX article.cst}
%TCIDATA{Created=Tuesday, June 10, 2014 22:31:12}
%TCIDATA{LastRevised=Wednesday, June 11, 2014 02:52:50}
%TCIDATA{<META NAME="GraphicsSave" CONTENT="32">}
%TCIDATA{<META NAME="SaveForMode" CONTENT="1">}
%TCIDATA{BibliographyScheme=Manual}
%TCIDATA{<META NAME="DocumentShell" CONTENT="Standard LaTeX\Blank - Standard LaTeX Article">}
%BeginMSIPreambleData
\providecommand{\U}[1]{\protect\rule{.1in}{.1in}}
%EndMSIPreambleData
\newtheorem{theorem}{Theorem}
\newtheorem{acknowledgement}[theorem]{Acknowledgement}
\newtheorem{algorithm}[theorem]{Algorithm}
\newtheorem{axiom}[theorem]{Axiom}
\newtheorem{case}[theorem]{Case}
\newtheorem{claim}[theorem]{Claim}
\newtheorem{conclusion}[theorem]{Conclusion}
\newtheorem{condition}[theorem]{Condition}
\newtheorem{conjecture}[theorem]{Conjecture}
\newtheorem{corollary}[theorem]{Corollary}
\newtheorem{criterion}[theorem]{Criterion}
\newtheorem{definition}[theorem]{Definition}
\newtheorem{example}[theorem]{Example}
\newtheorem{exercise}[theorem]{Exercise}
\newtheorem{lemma}[theorem]{Lemma}
\newtheorem{notation}[theorem]{Notation}
\newtheorem{problem}[theorem]{Problema}
\newtheorem{proposition}[theorem]{Proposition}
\newtheorem{remark}[theorem]{Remark}
\newtheorem{solution}[theorem]{Solution}
\newtheorem{summary}[theorem]{Summary}
\newenvironment{proof}[1][Proof]{\noindent\textbf{#1.} }{\ \rule{0.5em}{0.5em}}
\begin{document}
\section*{Setul 1}

\begin{problem}
\begin{enumerate}
\item[(a)] Fie o formul\u{a} de cuadratur\u{a} de tip Gauss de forma%
\[
\int\limits_{-a}^{a}w(t)f(t)\mathrm{d}\,t=A_{1}f(t_{1})+A_{2}f(t_{2})+R(f),
\]
unde $w$ este o func\c{t}ie par\u{a} (adic\u{a} $w(-t)=w(t)$, $\forall
t\in\lbrack-a,a]$). Ar\u{a}ta\c{t}i c\u{a} $t_{1}=-t_{2}$ \c{s}i $A_{1}=A_{2}$.

\item[(b)] Deduce\c{t}i o formul\u{a} de cuadratur\u{a} de forma
\[
\int\limits_{-\infty}^{\infty}e^{-|t|}f(t)\mathrm{d}\,t=A_{1}f(t_{1}%
)+A_{2}f(t_{2})+R(f),
\]
care s\u{a} aib\u{a} grad maxim de exactitate. Simplifica\c{t}i c\^{a}t mai
mult calculele folosind punctul (a).
\end{enumerate}
\end{problem}

\begin{problem}
Fie $a>0$. Pornind de la o ecua\c{t}ie convenabil\u{a} \c{s}i folosind metoda
lui Newton, deduce\c{t}i o metod\u{a} pentru aproximarea lui $\frac{1}%
{\sqrt{a}}$ f\u{a}r\u{a} \^{\i}mp\u{a}r\c{t}iri. Cum se alege valoarea de
pornire? Care este criteriul de oprire? Deduce\c{t}i de aici o metod\u{a}
pentru calculul lui $\sqrt{a}$ f\u{a}r\u{a} \^{\i}mp\u{a}r\c{t}iri.
\end{problem}

\vspace{3cm}

\section*{Setul 2}

\begin{problem}
\begin{enumerate}


\item[(a)] Fie o formul\u{a} de cuadratur\u{a} de tip Gauss de forma%
\[
\int\limits_{-a}^{a}w(t)f(t)\mathrm{d}\,t=A_{1}f(t_{1})+A_{2}f(t_{2})+R(f),
\]
unde $w$ este o func\c{t}ie par\u{a} (adic\u{a} $w(-t)=w(t)$, $\forall
t\in\lbrack-a,a]$). Ar\u{a}ta\c{t}i c\u{a} $t_{1}=-t_{2}$ \c{s}i $A_{1}=A_{2}$.

\item[(b)] Deduce\c{t}i o formul\u{a} de cuadratur\u{a} de forma
\[
\int\limits_{-1}^{1}\sqrt{|t|}f(t)\mathrm{d}\,t=A_{1}f(t_{1})+A_{2}%
f(t_{2})+R(f),
\]
care s\u{a} aib\u{a} grad maxim de exactitate. Simplifica\c{t}i c\^{a}t mai
mult calculele folosind punctul (a).
\end{enumerate}
\end{problem}

\begin{problem}
Fie $a>0$. Pornind de la o ecua\c{t}ie convenabil\u{a} \c{s}i folosind metoda
lui Newton, deduce\c{t}i o metod\u{a} pentru aproximarea lui $\frac{1}{a}$
f\u{a}r\u{a} \^{\i}mp\u{a}r\c{t}iri. Cum se alege valoarea de pornire? Care
este criteriul de oprire? Cum ve\c{t}i proceda pentru o implementare
eficient\u{a} \^{\i}n virgul\u{a} flotant\u{a}?
\end{problem}


\end{document}