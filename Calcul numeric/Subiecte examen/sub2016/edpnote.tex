\documentclass{book}%
\usepackage{amsfonts}
\usepackage{amsmath}
\usepackage{amssymb}
\usepackage{graphicx}%
\setcounter{MaxMatrixCols}{30}
%TCIDATA{OutputFilter=latex2.dll}
%TCIDATA{Version=5.50.0.2953}
%TCIDATA{CSTFile=40 LaTeX Book.cst}
%TCIDATA{Created=Wednesday, June 22, 2016 12:12:49}
%TCIDATA{LastRevised=Wednesday, June 22, 2016 15:11:06}
%TCIDATA{<META NAME="GraphicsSave" CONTENT="32">}
%TCIDATA{<META NAME="SaveForMode" CONTENT="1">}
%TCIDATA{BibliographyScheme=Manual}
%TCIDATA{<META NAME="DocumentShell" CONTENT="Standard LaTeX\Standard LaTeX Book">}
%BeginMSIPreambleData
\providecommand{\U}[1]{\protect\rule{.1in}{.1in}}
%EndMSIPreambleData
\newtheorem{theorem}{Theorem}
\newtheorem{acknowledgement}[theorem]{Acknowledgement}
\newtheorem{algorithm}[theorem]{Algorithm}
\newtheorem{axiom}[theorem]{Axiom}
\newtheorem{case}[theorem]{Case}
\newtheorem{claim}[theorem]{Claim}
\newtheorem{conclusion}[theorem]{Conclusion}
\newtheorem{condition}[theorem]{Condition}
\newtheorem{conjecture}[theorem]{Conjecture}
\newtheorem{corollary}[theorem]{Corollary}
\newtheorem{criterion}[theorem]{Criterion}
\newtheorem{definition}[theorem]{Definition}
\newtheorem{example}[theorem]{Example}
\newtheorem{exercise}[theorem]{Exercise}
\newtheorem{lemma}[theorem]{Lemma}
\newtheorem{notation}[theorem]{Notation}
\newtheorem{problem}[theorem]{Problem}
\newtheorem{proposition}[theorem]{Proposition}
\newtheorem{remark}[theorem]{Remark}
\newtheorem{solution}[theorem]{Solution}
\newtheorem{summary}[theorem]{Summary}
\newenvironment{proof}[1][Proof]{\noindent\textbf{#1.} }{\ \rule{0.5em}{0.5em}}
\begin{document}

\frontmatter
\title{Metode numerice pentru ecua\c{t}ii diferen\c{t}iale}
\author{Radu Tr\^{\i}mbi\c{t}a\c{s}}
\date{The Date}
\maketitle
\tableofcontents

\chapter*{Preface}

\markboth{PREFACE}{PREFACE}This is the preface. It is an unnumbered chapter.
The \verb|markboth| TeX field at the beginning of this paragraph sets the
correct page heading for the Preface portion of the document. The preface does
not appear in the table of contents.

\chapter{Introduction}

The introduction is entered using the usual chapter tag. Since the
introduction chapter appears before the \verb|mainmatter| TeX field, it is an
unnumbered chapter. The primary difference between the preface and the
introduction in this shell document is that the introduction will appear in
the table of contents and the page headings for the introduction are
automatically handled without the need for the \verb|markboth| TeX field. You
may use either or both methods to create chapters at the beginning of your
document. You may also delete these preliminary chapters.

\mainmatter


\part{Ecua\c{t}ii cu derivate par\c{t}iale}

\chapter{Introducere}

\begin{example}
Microprocesorul 8086 fabricat de Intel Corp. \^{\i}n anii 1970 rula la 5 MHz
\c{s}i necesita mai pu\c{t}in de 5 W putere. Ast\u{a}zi, cu vitez\u{a}
crescut\u{a} cu un factor de c\^{a}teva sute, microprocesoarele disipeaz\u{a}
peste 50 W. Pentru a evita deteriorarea procesorului din cauza temperaturii
excesive, este esen\c{t}ial s\u{a} se distribuie c\u{a}ldura utiliz\^{a}nd
radiatoare \c{s}i ventilatoare. Considera\c{t}iile de r\u{a}cire sunt un
obstacol constant \^{\i}n calea legii lui Moore referitoare la viteza de
prelucrare mai mare. Disiparea \^{\i}n timp a c\u{a}ldurii este modelat\u{a}
bine printr-o EDP parabolic\u{a}. C\^{a}nd c\u{a}ldura atinge un echilibru, o
ecua\c{t}ie eliptic\u{a} este mai potrivit\u{a} pentru modelarea
distribu\c{t}iei st\u{a}rii de echilibru. Aplica\c{t}ia practic\u{a} \ref{RC8}
de la pagina ??? ne arat\u{a} cum putem modela o configura\c{t}ie simpl\u{a}
de disipare a c\u{a}ldurii utiliz\^{a}nd o EDP eliptic\u{a} cu condi\c{t}ii
termice convective pe frontier\u{a}.
\end{example}

O EDP este o ecua\c{t}ie diferen\c{t}ial\u{a} cu mai mult de o variabil\u{a}
independent\u{a}. Domeniul fiind vast, vom limita discu\c{t}ia la ecua\c{t}ii
\^{\i}n dou\u{a} variabile independente, de forma
\begin{equation}
Au_{xx}+Bu_{xy}+Cu_{yy}+F(u_{x},u_{y},u,x,y)=0, \label{ec8.1}%
\end{equation}
unde derivatele par\c{t}iale sunt notate prin indicii $x$ \c{s}i $y$ pentru
variabilele independente, iar $u$ desemneaz\u{a} solu\c{t}ia. Dac\u{a} una din
variabile reprezint\u{a} timpul, ca \^{\i}n ecua\c{t}ia c\u{a}ldurii, vom nota
variabilele independente cu $x$ \c{s}i $t$.

\^{I}n fubc\c{t}ie de termenul dominant din (\ref{ec8.1}), solu\c{t}iile pot
avea propriet\u{a}\c{t}i diverse. EDP de ordinul al doilea cu dou\u{a}
variabile independente se clasific\u{a} astfel:

\begin{enumerate}
\item parabolice, dac\u{a} $B^{2}-4AC=0$

\item hiperbolice, dac\u{a} $B^{2}-4AC>0$

\item eliptice, dac\u{a} $B^{2}-4AC<0$
\end{enumerate}

Diferen\c{t}a practic\u{a} este c\u{a} ecua\c{t}iile parabolice \c{s}i
hiperbolice sunt definite \^{\i}ntr-o regiune deschis\u{a}. Condi\c{t}iile pe
frontier\u{a} pentru o variabil\u{a}---\^{\i}n cele mai multe cazuri
timpul---sunt specificate la un cap\u{a}t al regiunii, iar solu\c{t}ia se
ob\c{t}ine dep\u{a}rt\^{a}ndu-ne de acea frontier\u{a}. Ecua\c{t}iile
eliptice, pe de alt\u{a} parte, se specific\u{a} d\^{a}nd condi\c{t}ii pe
\^{\i}ntreaga frontier\u{a} a unei regiuni \^{\i}nchise. Vom studia exemple de
fiecare tip \c{s}i vom ilustra metodele numerice de rezolvare.

\chapter{Ecua\c{t}ii parabolice}

The \emph{heat equation}%
\begin{equation}
u_{t}=Du_{xx} \label{ec8.2}%
\end{equation}
represents temperature $x$ measured along a one-dimensional homogeneous
rod.The constant $D>0$ is called the \emph{diffusion coefficient},
representing the thermal diffusivity of the material making up the rod. The
heat equation models the spread of heat from regions of higher concentration
to regions of lower concentration. The independent variables are $x$ and $t$.

We use the variable $t$ instead of $y$ in (\ref{ec8.2}) because it represents
time. From the foregoing classification, we have $B^{2}-4AC=0$, so the heat
equation is parabolic. The so-called heat equation is an example of a
\emph{diffusion equation}, which models the diffusion of a substance. In
materials science, the same equation is known as Fick's second law and
describes diffusion of a substance within a medium.

Similar to the case of ODEs, the PDE (\ref{ec8.2}) has infinitely many
solutions, and extra conditions are needed to pin down a particular solution.
Chapters 6 and 7 treated the solution of ODEs, where initial conditions or
boundary conditions were used, respectively. In order to properly pose a PDE,
various combinations of initial and boundary conditions can be used.

For the heat equation, a straightforward analysis suggests which conditions
should be required. To specify the situation uniquely, we need to know the
initial temperature distribution along the rod and what is happening at the
ends of the rod as time progresses.

The properly posed heat equation on a finite interval has the form%
\begin{equation}
\left\{
\begin{array}
[c]{ll}%
u_{t}=Du_{xx}, & \forall t\in\lbrack a,b],t\geq0\\
u(x,0)=f(x), & \forall t\in\lbrack a,b]\\
u(a,t)=l(t), & \forall t\geq0\\
u(b,t)=r(t), & \forall t\geq0
\end{array}
\right.  \label{ec8.3}%
\end{equation}
where the rod extends along the interval $a\leq x\leq b$. The diffusion
coefficient $D$ governs the rate of heat transfer. The function $f(x)$ on
$[a,b]$ gives the initial temperature distribution along the rod, and $l(t)$,
$r(t)$ for $t\geq0$ give the temperature at the ends. Here, we have used a
combination of initial conditions $f(x)$ and boundary conditions $l(t)$ and
$r(t)$ to specify a unique solution of the PDE.

\section{Metoda diferen\c{t}elor progresive(Forward Difference Method)}

The use of finite difference methods to approximate the solution of a partial
differential equation follows the direction established in the previous two
chapters. The idea is to lay down a grid in the independent variables and
discretize the PDE. The continuous problem is changed into a discrete problem
of a finite number of equations. If the PDE is linear, the discrete equations
are linear and can be solved by the methods of Chapter 2.

To discretize the heat equation on the time interval $[0,T]$, we consider a
grid, or mesh, of points as shown in Figure \ref{fig8.1}. The closed circles
represent values of the solution $u(x,t)$ already known from the initial and
boundary conditions, and the open circles are mesh points that will be filled
in by the method.We will denote the exact solution by $u(x_{i},t_{j})$ and its
approximation at $(x_{i},t_{j})$ by $w_{ij}$. Let $M$ and $N$ be the total
number of steps in the $x$ and $t$ directions, and let $h=(b-a)/M$ and $k=T/N$
be the step sizes in the $x$ and $t$ directions.%

%TCIMACRO{\FRAME{ftbphFU}{6.9962cm}{5.0522cm}{0pt}{\Qcb{Mesh for the Finite
%Difference Method. The filled circles represent known initial and boundary
%conditions. The open circles represent unknown values that must be
%determined.}}{\Qlb{fig8.1}}{fig8_1mesh.png}%
%{\special{ language "Scientific Word";  type "GRAPHIC";
%maintain-aspect-ratio TRUE;  display "USEDEF";  valid_file "F";
%width 6.9962cm;  height 5.0522cm;  depth 0pt;  original-width 3.6357in;
%original-height 2.6143in;  cropleft "0";  croptop "1";  cropright "1";
%cropbottom "0";
%filename '../../../carti/carteanum/subspeciale/EDP/fig8_1mesh.png';file-properties "XNPEU";}%
%}}%
%BeginExpansion
\begin{figure}
[ptbh]
\begin{center}
\includegraphics[
natheight=2.614300in,
natwidth=3.635700in,
height=5.0522cm,
width=6.9962cm
]%
{../../../carti/carteanum/subspeciale/EDP/fig8_1mesh.png}%
\caption{Mesh for the Finite Difference Method. The filled circles represent
known initial and boundary conditions. The open circles represent unknown
values that must be determined.}%
\label{fig8.1}%
\end{center}
\end{figure}
%EndExpansion


The discretization formulas from Chapter 5 can be used to approximate
derivatives in the $x$ and $t$ directions. For example, applying the
centered-difference formula for the second derivative to the $x$ variable
yields
\begin{equation}
u_{xx}(x,t)\approx\frac{1}{h^{2}}\left(  u(x+h,t)-2u(x,t)+u(x-h,t)\right)
,\label{ec8.4}%
\end{equation}
with error $h^{2}u_{xxxx}(c_{1},t)/12$; and the forward-difference formula for
the first derivative used for the time variable gives%
\begin{equation}
u_{t}(x,t)\approx\frac{1}{k}\left(  u(x,t+k)-u(x,t)\right)  ,\label{ec8.5}%
\end{equation}
with error $ku_{tt}(x,c_{2})/2$, where $x-h<c_{1}<x+h$ and $t<c_{2}<t+h$.
Substituting into the heat equation at the point $(x_{i},t_{j})$ yields%
\begin{equation}
\frac{D}{h^{2}}\left(  w_{i+1,j}-2w_{ij}+w_{i-1,j}\right)  \approx\frac{1}%
{k}\left(  w_{i,j+1}-w_{ij}\right)  \label{ec8.6}%
\end{equation}
with the local truncation errors given by $O(k)+O(h^{2})$. Just as in our
study of ordinary differential equations, the local truncation errors will
give a good picture of the total errors, as long as the method is stable. We
will investigate the stability of the Finite Difference Method after
presenting the implementation details.

\appendix


\chapter{The First Appendix}

The appendix fragment is used only once. Subsequent appendices can be created
using the Chapter Section/Body Tag.

\backmatter


\chapter{Afterword}

The back matter often includes one or more of an index, an afterword,
acknowledgements, a bibliography, a colophon, or any other similar item. In
the back matter, chapters do not produce a chapter number, but they are
entered in the table of contents. If you are not using anything in the back
matter, you can delete the back matter TeX field and everything that follows it.
\end{document}