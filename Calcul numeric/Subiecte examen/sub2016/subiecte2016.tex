\documentclass{article}%
\usepackage{amsmath}
\usepackage{amsfonts}
\usepackage{amssymb}
\usepackage{graphicx}%
\setcounter{MaxMatrixCols}{30}
%TCIDATA{OutputFilter=latex2.dll}
%TCIDATA{Version=5.50.0.2953}
%TCIDATA{CSTFile=40 LaTeX article.cst}
%TCIDATA{Created=Thursday, May 19, 2016 00:31:09}
%TCIDATA{LastRevised=Thursday, May 19, 2016 00:58:42}
%TCIDATA{<META NAME="GraphicsSave" CONTENT="32">}
%TCIDATA{<META NAME="SaveForMode" CONTENT="1">}
%TCIDATA{BibliographyScheme=Manual}
%TCIDATA{<META NAME="DocumentShell" CONTENT="Standard LaTeX\Blank - Standard LaTeX Article">}
%BeginMSIPreambleData
\providecommand{\U}[1]{\protect\rule{.1in}{.1in}}
%EndMSIPreambleData
\newtheorem{theorem}{Theorem}
\newtheorem{acknowledgement}[theorem]{Acknowledgement}
\newtheorem{algorithm}[theorem]{Algorithm}
\newtheorem{axiom}[theorem]{Axiom}
\newtheorem{case}[theorem]{Case}
\newtheorem{claim}[theorem]{Claim}
\newtheorem{conclusion}[theorem]{Conclusion}
\newtheorem{condition}[theorem]{Condition}
\newtheorem{conjecture}[theorem]{Conjecture}
\newtheorem{corollary}[theorem]{Corollary}
\newtheorem{criterion}[theorem]{Criterion}
\newtheorem{definition}[theorem]{Definition}
\newtheorem{example}[theorem]{Example}
\newtheorem{exercise}[theorem]{Exercise}
\newtheorem{lemma}[theorem]{Lemma}
\newtheorem{notation}[theorem]{Notation}
\newtheorem{problem}[theorem]{Problema}
\newtheorem{proposition}[theorem]{Proposition}
\newtheorem{remark}[theorem]{Remark}
\newtheorem{solution}[theorem]{Solution}
\newtheorem{summary}[theorem]{Summary}
\newenvironment{proof}[1][Proof]{\noindent\textbf{#1.} }{\ \rule{0.5em}{0.5em}}
\begin{document}
\section*{Setul 1}

\begin{problem}
Pentru func\c{t}ia $f:\mathbb{R\rightarrow R}$, $f(x)=\cos\frac{\pi}{2}x$
\c{s}i diviziunea $\Delta:x_{1}=-1<x_{2}=0<x_{3}=1$, determina\c{t}i spline-ul
natural de interpolare.
\end{problem}

\begin{problem}
Fie $a>0$. Pornind de la o ecua\c{t}ie convenabil\u{a} \c{s}i folosind metoda
lui Newton, deduce\c{t}i o metod\u{a} pentru aproximarea lui $\frac{1}%
{\sqrt{a}}$ f\u{a}r\u{a} \^{\i}mp\u{a}r\c{t}iri. Cum se alege valoarea de
pornire? Care este criteriul de oprire? Deduce\c{t}i de aici o metod\u{a}
pentru calculul lui $\sqrt{a}$ f\u{a}r\u{a} \^{\i}mp\u{a}r\c{t}iri.
\end{problem}

\vspace{3cm}

\section*{Setul 2}

\begin{problem}
Pentru func\c{t}ia $f:\mathbb{R\rightarrow R}$, $f(x)=\sin\frac{\pi}{2}x$
\c{s}i diviziunea $\Delta:x_{1}=-1<x_{2}=0<x_{3}=1$, determina\c{t}i spline-ul
complet de interpolare.
\end{problem}

\begin{problem}
Fie $a>0$. Pornind de la o ecua\c{t}ie convenabil\u{a} \c{s}i folosind metoda
lui Newton, deduce\c{t}i o metod\u{a} pentru aproximarea lui $\frac{1}{a}$
f\u{a}r\u{a} \^{\i}mp\u{a}r\c{t}iri. Cum se alege valoarea de pornire? Care
este criteriul de oprire? Cum ve\c{t}i proceda pentru o implementare
eficient\u{a} \^{\i}n virgul\u{a} flotant\u{a}?
\end{problem}

\newpage

\section*{Setul 3}

\begin{problem}
Se consider\u{a} ecua\c{t}ia $f(x)=xe^{x}-1=0$. Dorim s\u{a} o rezolv\u{a}m
aplic\^{a}nd metoda aproxima\c{t}iilor succesive, rezolv\^{a}nd problema de
punct fix $x=F(x)$ \^{\i}n dou\u{a} moduri

\begin{enumerate}
\item[(a)] $F(x)=e^{-x}$

\item[(b)] $F(x)=\frac{1+x}{1+e^{x}}$.
\end{enumerate}

Ar\u{a}ta\c{t}i c\u{a} \^{\i}n ambele cazuri itera\c{t}iile $x_{k}=F(x_{k})$
sunt convergente, determina\c{t}i ordinul de convergen\c{t}\u{a} \c{s}i
num\u{a}rul de itera\c{t}ii necesare pentru a ob\c{t}ine precizia
$\varepsilon=10^{-10}$.
\end{problem}

\begin{problem}
Fie $f\in C^{4}[-1,1]$. Determina\c{t}i un polinom de interpolare $P$ de grad
minim care verific\u{a} condi\c{t}iile
\[
P(-1)=f(-1),~P^{\prime}(-1)=f^{\prime}(-1),~P(0)=f(0),~P(1)=f(1)
\]
\c{s}i determina\c{t}i expresia restului.
\end{problem}

\vspace{3cm}

\section*{Setul 4}

\begin{problem}
Pentru a rezolva ecua\c{t}ia $f(x)=0$ se aplic\u{a} metoda lui Newton
func\c{t}iei $g(x)=\frac{f(x)}{\sqrt{f^{\prime}(x)}}$.

\begin{enumerate}
\item[(a)] Scrie\c{t}i formula iterativ\u{a} care se ob\c{t}ine \c{s}i
determina\c{t}i ordinul de convergen\c{t}\u{a}.

\item[(b)] Aplica\c{t}i metoda de la punctul (a) pentru a aproxima $\sqrt{a}$,
$a>0$.
\end{enumerate}
\end{problem}

\begin{problem}
Fie $f\in C^{4}[-1,1]$. Determina\c{t}i un polinom de interpolare $P$ de grad
minim care verific\u{a} condi\c{t}iile
\[
P(-1)=f(-1),~~P(0)=f(0),~P(1)=f(1),P^{\prime}(1)=f^{\prime}(1).
\]
\c{s}i determina\c{t}i expresia restului.
\end{problem}


\end{document}