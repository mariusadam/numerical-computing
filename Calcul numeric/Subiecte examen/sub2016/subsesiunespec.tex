\documentclass{article}%
\usepackage{amsmath}
\usepackage{amsfonts}
\usepackage{amssymb}
\usepackage{graphicx}%
\setcounter{MaxMatrixCols}{30}
%TCIDATA{OutputFilter=latex2.dll}
%TCIDATA{Version=5.50.0.2953}
%TCIDATA{CSTFile=40 LaTeX article.cst}
%TCIDATA{Created=Wednesday, June 15, 2016 09:30:58}
%TCIDATA{LastRevised=Wednesday, June 15, 2016 09:42:58}
%TCIDATA{<META NAME="GraphicsSave" CONTENT="32">}
%TCIDATA{<META NAME="SaveForMode" CONTENT="1">}
%TCIDATA{BibliographyScheme=Manual}
%TCIDATA{<META NAME="DocumentShell" CONTENT="Standard LaTeX\Blank - Standard LaTeX Article">}
%BeginMSIPreambleData
\providecommand{\U}[1]{\protect\rule{.1in}{.1in}}
%EndMSIPreambleData
\newtheorem{theorem}{Theorem}
\newtheorem{acknowledgement}[theorem]{Acknowledgement}
\newtheorem{algorithm}[theorem]{Algorithm}
\newtheorem{axiom}[theorem]{Axiom}
\newtheorem{case}[theorem]{Case}
\newtheorem{claim}[theorem]{Claim}
\newtheorem{conclusion}[theorem]{Conclusion}
\newtheorem{condition}[theorem]{Condition}
\newtheorem{conjecture}[theorem]{Conjecture}
\newtheorem{corollary}[theorem]{Corollary}
\newtheorem{criterion}[theorem]{Criterion}
\newtheorem{definition}[theorem]{Definition}
\newtheorem{example}[theorem]{Example}
\newtheorem{exercise}[theorem]{Exercise}
\newtheorem{lemma}[theorem]{Lemma}
\newtheorem{notation}[theorem]{Notation}
\newtheorem{problem}[theorem]{Problem}
\newtheorem{proposition}[theorem]{Proposition}
\newtheorem{remark}[theorem]{Remark}
\newtheorem{solution}[theorem]{Solution}
\newtheorem{summary}[theorem]{Summary}
\newenvironment{proof}[1][Proof]{\noindent\textbf{#1.} }{\ \rule{0.5em}{0.5em}}
\begin{document}

\begin{enumerate}
\item[\textbf{P1}.] S\u{a} se stabileasc\u{a} formulele%
\begin{align*}
f^{\prime}(x)  &  =\frac{f(x+h)-f(x-h)}{2h}+O(h^{2})\\
f^{\prime\prime}(x)  &  =\frac{f(x+h)-2f(x)+f(x-h)}{h^{2}}+O(h^{2})
\end{align*}
deriv\^{a}nd formula de interpolare a lui Lagrange.

\item[\textbf{P2}.] 

\begin{enumerate}
\item[(a)] Consider\u{a}m itera\c{t}ia $x_{n+1}=F(x_{n})$ cu punctul fix
$\alpha$, care diverge deoarece $|F^{\prime}(\alpha)|>1$. Ar\u{a}ta\c{t}i
\ c\u{a} itera\c{t}ia%
\[
x_{n+1}=F^{-1}(x_{n})
\]
converge c\u{a}tre $\alpha$.

\item[(b)] Aplica\c{t}i rezultatul de mai sus pentru a calcula cea mai
mic\u{a} r\u{a}d\u{a}cin\u{a} pozitiv\u{a} a ecua\c{t}iei $x-\tan x=0.$
\end{enumerate}
\end{enumerate}

\bigskip\vspace{3cm}

\begin{enumerate}
\item[\textbf{P1}.] S\u{a} se stabileasc\u{a} formulele%
\begin{align*}
f^{\prime}(x)  &  =\frac{f(x+h)-f(x-h)}{2h}+O(h^{2})\\
f^{\prime\prime}(x)  &  =\frac{f(x+h)-2f(x)+f(x-h)}{h^{2}}+O(h^{2})
\end{align*}
deriv\^{a}nd formula de interpolare a lui Lagrange.

\item[\textbf{P2}.] 

\begin{enumerate}
\item[(a)] Consider\u{a}m itera\c{t}ia $x_{n+1}=F(x_{n})$ cu punctul fix
$\alpha$, care diverge deoarece $|F^{\prime}(\alpha)|>1$. Ar\u{a}ta\c{t}i
\ c\u{a} itera\c{t}ia%
\[
x_{n+1}=F^{-1}(x_{n})
\]
converge c\u{a}tre $\alpha$.

\item[(b)] Aplica\c{t}i rezultatul de mai sus pentru a calcula cea mai
mic\u{a} r\u{a}d\u{a}cin\u{a} pozitiv\u{a} a ecua\c{t}iei $x-\tan x=0.$
\end{enumerate}
\end{enumerate}


\end{document}