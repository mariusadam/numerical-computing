\documentclass[12pt]{article}%
\usepackage{amsmath}
\usepackage{amsfonts}
\usepackage{amssymb}
\usepackage{graphicx}%
\setcounter{MaxMatrixCols}{30}
%TCIDATA{OutputFilter=latex2.dll}
%TCIDATA{Version=5.50.0.2953}
%TCIDATA{CSTFile=40 LaTeX article.cst}
%TCIDATA{Created=Wednesday, May 31, 2017 13:26:34}
%TCIDATA{LastRevised=Wednesday, June 07, 2017 21:02:52}
%TCIDATA{<META NAME="GraphicsSave" CONTENT="32">}
%TCIDATA{<META NAME="SaveForMode" CONTENT="1">}
%TCIDATA{BibliographyScheme=Manual}
%TCIDATA{<META NAME="DocumentShell" CONTENT="Standard LaTeX\Blank - Standard LaTeX Article">}
%BeginMSIPreambleData
\providecommand{\U}[1]{\protect\rule{.1in}{.1in}}
%EndMSIPreambleData
\newtheorem{theorem}{Theorem}
\newtheorem{acknowledgement}[theorem]{Acknowledgement}
\newtheorem{algorithm}[theorem]{Algorithm}
\newtheorem{axiom}[theorem]{Axiom}
\newtheorem{case}[theorem]{Case}
\newtheorem{claim}[theorem]{Claim}
\newtheorem{conclusion}[theorem]{Conclusion}
\newtheorem{condition}[theorem]{Condition}
\newtheorem{conjecture}[theorem]{Conjecture}
\newtheorem{corollary}[theorem]{Corollary}
\newtheorem{criterion}[theorem]{Criterion}
\newtheorem{definition}[theorem]{Definition}
\newtheorem{example}[theorem]{Example}
\newtheorem{exercise}[theorem]{Exercise}
\newtheorem{lemma}[theorem]{Lemma}
\newtheorem{notation}[theorem]{Notation}
\newtheorem{problem}[theorem]{Problema}
\newtheorem{proposition}[theorem]{Proposition}
\newtheorem{remark}[theorem]{Remark}
\newtheorem{solution}[theorem]{Solution}
\newtheorem{summary}[theorem]{Summary}
\newenvironment{proof}[1][Proof]{\noindent\textbf{#1.} }{\ \rule{0.5em}{0.5em}}
\begin{document}
\section{Subiectul 1}

\begin{problem}
\begin{enumerate}
\item[(a)] S\u{a} se determine o formul\u{a} de cuadratur\u{a} de forma%
\[
\int_{0}^{1}\frac{1}{\sqrt{x(1-x)}}f(x)\mathrm{d}\,x=\sum_{k=1}^{n}%
A_{k}f(x_{k})+R(f)
\]
care s\u{a} aib\u{a} grad maxim de exactitate. (4p)

\item[(b)] Folosind formula de la punctul (a), s\u{a} se scrie cod MATLAB care
calculeaz\u{a}%
\[
\int_{0}^{1}\frac{1}{\sqrt{x(1-x)}}\sin(\pi x)\mathrm{d}\,x
\]
cu o precizie dat\u{a}. (Se pot folosi func\c{t}ii de la laborator) (2p)
\end{enumerate}
\end{problem}

\begin{proof}
[Solu\c{t}ie]

\begin{enumerate}
\item[(a)] Efectu\^{a}nd schimbarea de variabil\u{a} $x=\frac{t+1}{2}$,
integrala din enun\c{t} devine%
\[
\int_{0}^{1}\frac{1}{\sqrt{x(1-x)}}f(x)\mathrm{d}\,x=\int_{-1}^{1}%
\frac{f\left(  \frac{t+1}{2}\right)  }{\sqrt{1-t^{2}}}\mathrm{d}\,t\text{.}%
\]
Vom folosi o formul\u{a} de tip Gauss cu ponderea $w(t)=\frac{1}{\sqrt
{1-t^{2}}}$. Formula va fi
\[
\int_{0}^{1}\frac{1}{\sqrt{x(1-x)}}f(x)\mathrm{d}\,x=\sum_{k=1}^{n}%
A_{k}f(x_{k})+R(f)
\]
cu
\[
A_{k}=\frac{\pi}{n},x_{k}=\frac{1+\cos\frac{2k-1}{2n}\pi}{2},
\]
iar restul%
\[
R_{n}(f)=\frac{f^{(2n)}(\xi)}{(2n)!}\int_{-1}^{1}T_{n}^{2}(t)\mathrm{d}%
\,t=\frac{1}{(2)^{2n}(2n)!}\frac{\pi}{2^{2n+1}}f^{(2n)}(\xi).
\]

\end{enumerate}
\end{proof}

\begin{problem}
\label{Gautschip4.19}Se consider\u{a} ecua\c{t}ia
\[
\tan x+\lambda x=0,\quad0<\lambda<1.
\]


\begin{enumerate}
\item[(a)] Ar\u{a}ta\c{t}i c\u{a} \^{\i}n intervalul $\left[  \frac{1}{2}%
\pi,\pi\right]  $, ecua\c{t}ia are exact o r\u{a}d\u{a}cin\u{a}, $\alpha$. (1p)

\item[(b)] Converge metoda lui Newton c\u{a}tre $\alpha\in\left[  \frac{1}%
{2}\pi,\pi\right]  $, dac\u{a} aproxima\c{t}ia ini\c{t}ial\u{a} este
$x_{0}=\pi$? Justifica\c{t}i r\u{a}spunsul. (2p)
\end{enumerate}
\end{problem}

\begin{proof}
[Solu\c{t}ie]

\begin{enumerate}
\item[(a)] Graficele lui $y=\tan x$ \c{s}i $y=-\lambda x$ pentru $x>0$ se
intersecteaz\u{a} \^{\i}ntr-un punct situat \^{\i}ntre $\frac{1}{2}\pi$ \c{s}i
$\pi$, a c\u{a}rui abscis\u{a} este r\u{a}d\u{a}cina ecua\c{t}iei. Este
singura r\u{a}d\u{a}cin\u{a} \^{\i}n acel interval.

\item[(b)] Pentru $f(x)=\tan x+x$, avem, pe $\left[  \frac{1}{2}\pi
,\pi\right]  $,%
\begin{align*}
f\left(  \frac{\pi}{2}+0\right)   &  =-\infty,~f\left(  \pi\right)
=\lambda\pi\\
f^{\prime}\left(  x\right)   &  =\frac{1}{\cos^{2}x}+\lambda>0,\\
f^{\prime\prime}(x) &  =\frac{d}{dx}\left(  \frac{1}{\cos^{2}x}\right)
=\frac{2}{\cos^{3}x}\sin x<0,
\end{align*}
deci  $f$ este cresc\u{a}toare \c{s}i concav\u{a}. Dac\u{a} metoda lui Newton
se aplic\u{a} pentru  $x_{0}=$ $\pi$, atunci \c{s}irul aproximantelor este
monoton cresc\u{a}tor dac\u{a} $x_{1}>\frac{\pi}{2}$. Dar,
\[
x_{1}=\pi-\frac{f\left(  \pi\right)  }{f^{\prime}\left(  \pi\right)  }%
=\frac{\pi}{1+\lambda}>\frac{\pi}{2},
\]
c\u{a}ci $\lambda\in\lbrack0,1]$.
\end{enumerate}
\end{proof}

\newpage

\section{Subiectul 2}

\begin{problem}


\begin{enumerate}
\item[(a)] S\u{a} se determine o formul\u{a} de cuadratur\u{a} de forma%
\[
\int_{0}^{1}\sqrt{x(1-x)}f(x)\mathrm{d}\,x=A_{1}f(x_{1})+A_{2}f(x_{2}%
)+A_{3}f(x_{3})+R(f)
\]
care s\u{a} aib\u{a} grad maxim de exactitate. (3p)

\item[(b)] Folosind formula de la punctul (a) pentru $n$ noduri, s\u{a} se
scrie cod MATLAB care calculeaz\u{a}%
\[
\int_{0}^{1}\sqrt{x(1-x)}\sin(\pi x)\mathrm{d}\,x
\]
cu o precizie dat\u{a}. (Se pot folosi func\c{t}ii de la laborator). (2p)
\end{enumerate}
\end{problem}

\begin{proof}
[Solu\c{t}ie]

\begin{enumerate}
\item[(a)] Efectu\^{a}nd schimbarea de variabil\u{a} $x=\frac{t+1}{2}$,
integrala din enun\c{t} devine%
\[
\int_{0}^{1}\sqrt{x(1-x)}f(x)\mathrm{d}\,x=\frac{1}{4}\int_{-1}^{1}%
\sqrt{1-t^{2}}f\left(  \frac{t+1}{2}\right)  \mathrm{d}\,t
\]
Este o cuadratur\u{a} Gauss-Ceb\^{\i}\c{s}ev de spe\c{t}a a doua. Polinomul
ortogonal este%
\[
\pi_{3}(x)=t^{3}-\frac{1}{2}t
\]
cu r\u{a}d\u{a}cinile $-\frac{1}{2}\sqrt{2}$, $0$, $\frac{1}{2}\sqrt{2}$
($t_{k}=\cos\frac{k\pi}{n+1}$, $k=1,3$). Nodurile vor fi $t_{k}$, iar
coeficien\c{t}ii
\[
A_{1}=\frac{\pi}{8},A_{2}=\frac{\pi}{4},A_{3}=\frac{\pi}{8}.
\]
Restul
\[
R(f)=\frac{f^{(6)}(\xi)}{6!}\int_{-1}^{1}\sqrt{1-t^{2}}\left(  t^{3}-\frac
{1}{2}t\right)  ^{2}\mathrm{d}\,t=\frac{\pi}{92\,160}f^{(6)}(\xi)
\]
Revenind la substitu\c{t}ie, avem%
\begin{align*}
\int_{0}^{1}\sqrt{x(1-x)}f(x)\mathrm{d}\,x  & =\frac{1}{4}\left[  \frac{\pi
}{8}f\left(  \frac{1-\frac{\sqrt{2}}{2}}{2}\right)  +\frac{\pi}{4}f\left(
\frac{1}{2}\right)  +\frac{\pi}{8}f\left(  \frac{1+\frac{\sqrt{2}}{2}}%
{2}\right)  \right.  \\
& \left.  +\frac{\pi}{92\,160\cdot2^{6}}f^{(6)}(\zeta)\right]
\end{align*}

\end{enumerate}
\end{proof}

\begin{problem}
Dac\u{a} $A>0$, atunci $\alpha=\sqrt{A}$ este r\u{a}d\u{a}cin\u{a} a
ecua\c{t}iilor
\[
x^{2}-A=0,\qquad\frac{A}{x^{2}}-1=0.
\]


\begin{enumerate}
\item[(a)] Explica\c{t}i de ce metoda lui Newton aplicat\u{a} primei
ecua\c{t}ii converge pentru o valoare de pornire arbitrar\u{a} $x_{0}>0$.(1p)

\item[(b)] Explica\c{t}i de ce metoda lui Newton aplicat\u{a} la a doua
ecua\c{t}ie produce iterate pozitive $(x_{n})$ ce converg la $\alpha$ numai
dac\u{a} $x_{0}$ este situat \^{\i}ntr-un interval $0<x_{0}<b$.
Determina\c{t}i $b$. (2p)

\item[(c)] Descrie\c{t}i \^{\i}n fiecare caz algoritmul (itera\c{t}ia,
criteriul de oprire, valoarea de pornire).(1p)
\end{enumerate}
\end{problem}

\begin{proof}
[Solu\c{t}ie]

\begin{enumerate}
\item[(a)] \^{I}n primul caz,  $f(x)=x^{2}-A$ este convex\u{a} pe
$\mathbb{R}_{+}$ \c{s}i cresc\u{a}toare pe $(0,\infty)$. Newton coverge pentru
orice $x_{0}>0$. Altfel: dac\u{a} $x_{0}>\alpha$, atunci $\left(
x_{n}\right)  $ converge monoton descresc\u{a}tor c\u{a}tre $\alpha$. Dac\u{a}
$0<x_{0}<\alpha$, atunci $x_{1}>\alpha$, \c{s}i se ra\c{t}ioneaz\u{a} la fel.,
pentru $n>1$. 

\item[(b)] \^{I}n al doilea caz, $f(x)=$ $\frac{A}{x^{2}}-1$ este convex\u{a}
$\mathbb{R}_{+}$ \c{s}i descrec\u{a}toare ($\lim_{x\searrow0}f(x)=\infty$ to
$\lim_{x\rightarrow\infty}f(x)=-1$). Dac\u{a} $0<x_{0}<\alpha$, atunci
$\left(  x_{n}\right)  $ converge  monoton cresc\u{a}tor c\u{a}tre $\alpha$.
Dac\u{a} $x_{0}>\alpha$, trebui s\u{a} ne asigur\u{a}m c\u{a} $x_{1}>0$, ceea
ce \^{\i}nseamn\u{a} c\u{a}
\begin{align*}
x_{1} &  =x_{0}-\frac{\frac{A}{x_{0}^{2}}-1}{-2\frac{A}{x_{0}^{3}}}>0,\quad
x_{0}+x_{0}\frac{A-x_{0}^{2}}{2A}>0\\
x_{0}\left(  3A-x_{0}^{2}\right)   &  >0,~x_{0}<\sqrt{3A}=:b.
\end{align*}


\item[(c)] Se alege $x_{0}\in(0,b)$, itera\c{t}ia
\[
x_{n+1}=x_{n}+x_{n}\frac{A-x_{0}^{2}}{2A}%
\]
criteriul%
\[
|x_{n+1}-x_{n}|<\varepsilon.
\]

\end{enumerate}
\end{proof}


\end{document}