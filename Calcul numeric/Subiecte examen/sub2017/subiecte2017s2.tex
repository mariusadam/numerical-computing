\documentclass{article}%
\usepackage{amsmath}
\usepackage{amsfonts}
\usepackage{amssymb}
\usepackage{graphicx}%
\setcounter{MaxMatrixCols}{30}
%TCIDATA{OutputFilter=latex2.dll}
%TCIDATA{Version=5.50.0.2953}
%TCIDATA{CSTFile=40 LaTeX article.cst}
%TCIDATA{Created=Monday, June 05, 2017 23:41:38}
%TCIDATA{LastRevised=Thursday, June 08, 2017 15:09:56}
%TCIDATA{<META NAME="GraphicsSave" CONTENT="32">}
%TCIDATA{<META NAME="SaveForMode" CONTENT="1">}
%TCIDATA{BibliographyScheme=Manual}
%TCIDATA{<META NAME="DocumentShell" CONTENT="Standard LaTeX\Blank - Standard LaTeX Article">}
%BeginMSIPreambleData
\providecommand{\U}[1]{\protect\rule{.1in}{.1in}}
%EndMSIPreambleData
\newtheorem{theorem}{Theorem}
\newtheorem{acknowledgement}[theorem]{Acknowledgement}
\newtheorem{algorithm}[theorem]{Algorithm}
\newtheorem{axiom}[theorem]{Axiom}
\newtheorem{case}[theorem]{Case}
\newtheorem{claim}[theorem]{Claim}
\newtheorem{conclusion}[theorem]{Conclusion}
\newtheorem{condition}[theorem]{Condition}
\newtheorem{conjecture}[theorem]{Conjecture}
\newtheorem{corollary}[theorem]{Corollary}
\newtheorem{criterion}[theorem]{Criterion}
\newtheorem{definition}[theorem]{Definition}
\newtheorem{example}[theorem]{Example}
\newtheorem{exercise}[theorem]{Exercise}
\newtheorem{lemma}[theorem]{Lemma}
\newtheorem{notation}[theorem]{Notation}
\newtheorem{problem}[theorem]{Problema}
\newtheorem{proposition}[theorem]{Proposition}
\newtheorem{remark}[theorem]{Remark}
\newtheorem{solution}[theorem]{Solution}
\newtheorem{summary}[theorem]{Summary}
\newenvironment{proof}[1][Proof]{\noindent\textbf{#1.} }{\ \rule{0.5em}{0.5em}}
\begin{document}
\section*{Subiectul 3}

\begin{problem}
\label{Gautschip3.34}Consider\u{a}m problema de interpolare Hermite:
determina\c{t}i $H_{2n-1}f\in P_{2n-1}$ astfel \^{\i}nc\^{a}t
\begin{equation}
\left(  H_{2n-1}f\right)  \left(  \tau_{\nu}\right)  =f\left(  \tau_{\nu
}\right)  ,~\left(  H_{2n-1}f\right)  ^{\prime}\left(  \tau_{\nu}\right)
=f^{\prime}\left(  \tau_{\nu}\right)  ,~\nu=1,2,\dots,n. \tag{(*)}%
\label{fstea}%
\end{equation}
Prin analogie cu formula lui Lagrange, polinomul care rezolv\u{a} \ref{fstea}
se poate scrie cu ajutorul polinoamelor fundamentale de interpolare
Hermite\ $h_{\nu,0}$, $h_{\nu,1}$ sub forma%
\[
\left(  H_{2n-1}f\right)  (t)=\sum_{\nu=1}^{n}\left[  h_{\nu,0}(t)f_{\nu
}+h_{\nu,1}\left(  t\right)  f_{\nu}^{\prime}\right]  .
\]


\begin{enumerate}
\item[(a)] C\u{a}uta\c{t}i $h_{\nu,0}$ \c{s}i $h_{\nu,1}\left(  t\right)  $
sub forma
\[
h_{\nu,0}\left(  t\right)  (t)=\left(  a_{\nu}+b_{\nu}t\right)  \ell_{\nu}%
^{2}(t),~h_{\nu,1}\left(  t\right)  \left(  t\right)  =\left(  c_{\nu}+d_{\nu
}t\right)  \ell_{\nu}^{2}(t),
\]
unde $\ell_{\nu}$ sunt polinoamele fundamentale de interpolare Lagrange.
Determina\c{t}i constantele $a_{\nu}$, $b_{\nu}$, $c_{\nu}$, $d_{\nu}$.

\item[(b)] Ob\c{t}ine\c{t}i formula de cuadratur\u{a}
\[
\int_{a}^{b}f(t)w(t)\mathrm{d}\,t=\sum_{\nu=1}^{n}\left[  \lambda_{\nu
}f\left(  \tau_{\nu}\right)  +\mu_{\nu}f^{\prime}\left(  \tau_{\nu}\right)
\right]  +R_{n}(f)
\]
cu proprietatea $R_{n}\left(  f\right)  =0$ pentru orice $f\in\mathbb{P}%
_{2n-1}$.

\item[(c)] Ce condi\c{t}ii trebuie impuse asupra polinomului nodurilor
$\omega_{n}(t)=\prod\limits_{\nu=1}^{n}\left(  t-t_{\nu}\right)  $ (sau asupra
nodurilor $\tau_{\nu}$) astfel ca $\mu_{\nu}=0$ pentru $\nu=1,2,\dots,n$?
\end{enumerate}
\end{problem}

\begin{proof}
[Solu\c{t}ie]

\begin{enumerate}
\item[(a)] Polinoamele $h_{\nu,0}$ trebuie s\u{a} verifice%
\[
h_{\nu,0}\left(  \tau_{\nu}\right)  =1,~h_{\nu,0}^{\prime}\left(  \tau_{\nu
}\right)  =0,
\]
iar condi\c{t}iile  $h_{\nu,0}(\tau_{%
%TCIMACRO{\U{3bc} }%
%BeginExpansion
\mu
%EndExpansion
})=h_{\nu,0}^{\prime}(\tau_{%
%TCIMACRO{\U{3bc} }%
%BeginExpansion
\mu
%EndExpansion
})=0$, $\mu\neq\nu$, sunt automat verificate datorit\u{a} formei lui
$h_{\nu,0}$. Astfel,%
\[
a_{\nu}+b_{\nu}\tau_{\nu}=1,~b_{\nu}+\left(  a_{\nu}+b_{\nu}\tau_{\nu}\right)
\cdot2\ell_{\nu}\left(  \tau_{\nu}\right)  \ell_{\nu}^{\prime}\left(
\tau_{\nu}\right)  =0,
\]
adic\u{a},%
\begin{align*}
a_{\nu}+b_{\nu}\tau_{\nu} &  =1,\\
b_{\nu}+2\ell_{\nu}^{\prime}\left(  \tau_{\nu}\right)   &  =0.
\end{align*}
Rezolv\^{a}nd sistemul cu necunoscutele $a_{\nu}$ \c{s}i  $b_{\nu}$ \c{s}i
\^{\i}nlocuind \^{\i}n  $h_{\nu,0}$ se ob\c{t}ine%
\[
h_{\nu,0}(t)=\left[  1-2(t-\tau_{\nu})\ell_{\nu}^{\prime}(\tau_{\nu})\right]
\ell_{\nu}^{2}(t),\quad\nu=1,2,\dots,n.
\]
Analog, $h_{\nu,1}$ satisface%
\[
h_{\nu,1}\left(  \tau_{\nu}\right)  =0,~h_{\nu,1}^{\prime}\left(  \tau_{\nu
}\right)  =1
\]
din care se ob\c{t}ine
\[
c_{\nu}+d_{\nu}\tau_{\nu}=0,~d_{\nu}+(c_{\nu}+d_{\nu}\tau_{\nu})\cdot
2\ell_{\nu}(\tau_{\nu})\ell_{\nu}^{\prime}(\tau_{\nu})=1,
\]
adic\u{a},
\[
c_{\nu}+d_{\nu}\tau_{\nu}=0,\quad d_{\nu}=1.
\]
Astfel, $c_{\nu}=-\tau_{\nu}$ \c{s}i\
\[
h_{\nu,1}(t)=(t-\tau_{\nu})\ell_{\nu}^{2}(t),\quad\nu=1,2,\dots,n.
\]
Derivata polinomului fundamental \^{\i}n $\tau_{\nu}$ este%
\[
\ell_{\nu}^{\prime}(\tau_{\nu})=\sum_{\mu\neq\nu}\frac{1}{\tau_{\nu}-\tau
_{\mu}}.
\]


\item[(b)] Formula de cuadratur\u{a} se ob\c{t}ine integr\^{a}nd termen cu
termen%
\[
\int_{a}^{b}f(t)w(t)\mathrm{d}\,t=\int_{a}^{b}p(t)w(t)\mathrm{d}\,t+R_{n}(f),
\]
Gradul de exactitate este $2n-1$. Utiliz\^{a}nd punctul (a), se ob\c{t}ine%
\begin{align*}
\int_{a}^{b}p(t)w(t)\mathrm{d}\,t &  =\int_{a}^{b}\sum_{\nu=1}^{n}\left[
h_{\nu,0}(t)f_{\nu}+h_{\nu,1}\left(  t\right)  f_{\nu}^{\prime}\right]
w(t)\mathrm{d}\,t\\
&  =\sum_{\nu=1}^{n}\left[  f_{\nu}\int_{a}^{b}h_{\nu}(t)w(t)\mathrm{d}%
\,t+f_{\nu}^{\prime}\int_{a}^{b}k_{\nu}\left(  t\right)  w(t)\mathrm{d}%
\,t\right]  .
\end{align*}
Deci%
\[
\lambda_{\nu}=\int_{a}^{b}h_{\nu,0}(t)w(t)\mathrm{d}\,t,\quad\mu_{\nu}%
=\int_{a}^{b}h_{\nu,1}\left(  t\right)  w(t)\mathrm{d}\,t,\quad\nu
=1,2,\dots,n.
\]


\item[(b)] Pentru ca to\c{t}i coeficien\c{t}ii $\mu$ s\u{a} fie nuli, trebuie
s\u{a} avem
\[
\int_{a}^{b}h_{\nu,1}\left(  t\right)  w(t)\mathrm{d}\,t=0,\quad\nu
=1,2,\dots,n.
\]
sau, pe baza lui (a), observ\^{a}nd c\u{a} $\ell_{\nu}(t)=\frac{\omega_{n}%
(t)}{(t-\tau_{\nu})\omega_{n}^{\prime}(\tau_{\nu})}$,%
\[
\frac{1}{\omega_{n}^{\prime}(\tau_{\nu})}\int_{a}^{b}\frac{\omega_{n}%
(t)}{(t-\tau_{\nu})\omega_{n}^{\prime}(\tau_{\nu})}\omega_{n}(t)w(t)\mathrm{d}%
\,t=0,\quad\nu=1,2,\dots,n.
\]
adic\u{a},%
\[
\int_{a}^{b}\ell_{\nu}(t)\omega_{n}(t)w(t)\mathrm{d}\,t=0,\quad\nu
=1,2,\dots,n.
\]
Deoarece $\{\ell_{\nu}(t)\}_{\nu=1}^{n}$ formeaz\u{a} o baz\u{a} a lui
$\mathbb{P}_{n-1}$ ($\ell_{\nu}$ sunt liniar independente \c{s}i genereaz\u{a}
$\mathbb{P}_{n-1}$), $\omega_{n}$ trebuie s\u{a} fie ortogonal pe $[a,b]$
\^{\i}n raport cu $w(t)=1$ pe toate polinoamele de grad mai mic, adic\u{a},
$\omega_{n}(t)=\pi_{n}(t;w)$. Se ob\c{t}ine o formul\u{a} de cuadratur\u{a} gaussian\u{a}.
\end{enumerate}
\end{proof}

\begin{problem}
Implementa\c{t}i \^{\i}n MATLAB o metod\u{a} hibrid\u{a}
Newton+\^{\i}njum\u{a}t\u{a}\c{t}ire pentru rezolvarea ecua\c{t}iei $f(x)=0$,
$f\in C^{1}$. Testa\c{t}i pentru $f(x)=J_{0}(x)$, pe intervalul $[0,4]$ \c{s}i
compara\c{t}i cu metoda lui Newton pentru $x_{0}=0.01$.
\end{problem}


\begin{verbatim}
function [xFinal,ni]=Newtonsafevb(f,fd,a,b,tol,nitmax,varargin)
% NEWTSAFEVB - root-finder using hybrid Newton-Bisection method to always maintain bracket
%
% [xFinal, xN, errorN, ni]=Newtonsafe(f,fd,a,b,tol,nitmax,varargin)
%
% f, fd - returns the function and its derivative
% a,b: initial bracket
% tol: stopping condition for error f(x) <= tol or |b-a|<tol*b
%
% xFinal: final value
% xN: vector of intermediate iterates
% errorN: vector of errors
if nargin<6, nitmax=50; end
% initialize the problem
h = b-a;
fa = f(a,varargin{:});
fb = f(b,varargin{:});
if ( sign(fa) == sign(fb) )
    error('function must be bracketed to begin with' );
end
c = a ; % start on the left side (could also choose the middle
fc = f(c,varargin{:}); df=fd(c,varargin{:}) ;
%xN(1) = c;
%errorN(1,:) = [ abs(fc) h ];
% begin iteration until convergence or Maximum Iterations
ni=1;
for i = 1:nitmax
    % try a Newton step
    c = c - fc/df;
    % if not in bracket choose bisection
    if ( ~(a <= c && b >= c) )
        c = a + h/2;
    end
    % Evaluate function and derivative at new c
    fc = f(c,varargin{:}); df=fd(c,varargin{:});
    % check and maintain bracket
    if ( sign(fc) ~= sign(fb) )
        a=c;
        fa=fc;
    else
        b = c;
        fb = fc;
    end
    h = b-a;
    % calculate errors and track solutions
    absError = abs(fc);
    relError = h;
    % check if converged
    if ( absError < tol || relError < tol*b ) %succes
        xFinal=c; ni=i; return;
    end
end
% clean up
error('Maximum iterations exceeded' );

\end{verbatim}

Test
\begin{verbatim}
%testsub3
g = @(x) besselj(0,x);
gd = @ (x) -besselj(1, x);
[z2,ni2]=Newton(g,gd,0.01,tol)
[z5, ni5]=Newtonsafevb(g,gd,0,4,tol)
\end{verbatim}
Rezultate:
\begin{verbatim}
z2 =
200.2772
ni2 =
5
z5 =
2.4048
ni5 =
4
\end{verbatim}

\newpage

\section*{Subiectul 4}

\begin{problem}
\label{Gautschip3.46}

\begin{enumerate}
\item[(a)] Utiliza\c{t}i metoda coeficien\c{t}ilor nedetermina\c{t}i pentru a
ob\c{t}ine formula de cuadratur\u{a} (cu gradul de exactitate $d\geq2$) de
forma
\[
\int_{0}^{1}y(s)\mathrm{d}\,s\approx ay(0)+by(1)-c\left[  y^{\prime
}(1)-y^{\prime}(0)\right]  +R(f).
\]


\item[(b)] Transforma\c{t}i formula de la (a) \^{\i}ntr-o formul\u{a} pentru a
aproxima $\int_{x}^{h+x}f(t)\mathrm{d}\,t$.

\item[(c)] Ob\c{t}ine\c{t}i o formul\u{a} de integrare repetat\u{a} bazat\u{a}
pe formula de la (b) pentru a aproxima $\int_{a}^{b}f(t)\mathrm{d}\,t$.
Interpreta\c{t}i rezultatul.
\end{enumerate}
\end{problem}

\begin{proof}
[Solu\c{t}ie]

\begin{enumerate}
\item[(a)] Pun\^{a}nd $y(s)=1$, $y(s)=s$, $y(s)=s^{2}$, din condi\c{t}iile de
exactitate se ob\c{t}ine%
\begin{align*}
a+b+0c &  =1\\
0a+b-0c &  =\frac{1}{2}\\
0a+b-2c &  =\frac{1}{3}%
\end{align*}
Solu\c{t}ia este: $a=\frac{1}{2},b=\frac{1}{2},c=\frac{1}{12}$, adic\u{a},%
\[
\int_{0}^{1}y(s)\mathrm{d}\,s=\frac{1}{2}\left[  y(0)+y(1)\right]  -\frac
{1}{12}\left[  y^{\prime}(1)-y^{\prime}(0)\right]  +R(f)
\]%
\[
R(f)=\frac{f^{(4)}(\xi)}{4!}\int_{0}^{1}s^{2}(s-1)^{2}\mathrm{d}\,s=\frac
{1}{720}f^{(4)}(\xi)
\]
: 

\item[(b)] Schimbarea de variabil\u{a} $t=x+hs$, $dt=hds$ ne conduce la%
\begin{align*}
\int_{x}^{x+h}f(t)\mathrm{d}\,t &  =h\int_{0}^{1}\left(  x+hs\right)
\mathrm{d}\,s=\frac{h}{2}\left[  f(x)+f(x+h)\right]  \\
&  -\frac{h^{2}}{12}\left[  f^{\prime}(x+h)-f^{\prime}(h)\right]  +\frac
{h^{4}}{720}f^{(4)}\left(  \zeta\right)  .
\end{align*}


\item[(c)] Pun\^{a}nd $h=(b-a)/n$, $x_{k}=a+kh$, $f_{k}=f(x_{k})$,
$f_{k}^{\prime}=f^{\prime}(x_{k})$, $k=0,1,\dots,n$, ob\c{t}inem folosind (b),
c\u{a}%
\begin{align*}
\int_{a}^{b}f(t)\mathrm{d}\,t &  =\sum_{k=0}^{n-1}\int_{x_{k}}^{x_{k}%
+h}f(t)\mathrm{d}\,t\approx\frac{h}{2}\left[  \left(  f_{0}+f_{1}\right)
+\left(  f_{1}+f_{2}\right)  +\cdots\right.  \\
&  +\left.  \left(  f_{n-1}+f_{n}\right)  \right]  -\frac{h^{2}}{12}\left[
\left(  f_{1}^{\prime}-f_{0}^{\prime}\right)  +\left(  f_{2}^{\prime}%
-f_{10}^{\prime}\right)  +\cdots+\left(  f_{n}^{\prime}-f_{n-1}^{\prime
}\right)  \right]  \\
&  =h\left(  \frac{1}{2}f_{0}+f_{1}+\cdots+f_{n-1}+\frac{1}{2}f_{n}\right)
-\frac{h^{2}}{12}\left[  f^{\prime}(b)-f^{\prime}(a)\right]  .
\end{align*}
Restul%
\[
R_{n}(f)=\frac{(b-a)^{4}}{720n^{3}}f^{(4)}(\zeta)
\]
Interpretare regula trapezelor cu o  \textquotedblleft corec\c{t}ie la
capete\textquotedblright. Corec\c{t}ia aproximeaz\u{a} eroarea \^{\i}n formula
trapezelor:%
\[
-\frac{b-a}{12}h^{2}f^{\prime\prime}(\xi)\approx-\frac{h^{2}}{12}\left[
f^{\prime}(b)-f^{\prime}(a)\right]  .
\]

\end{enumerate}
\end{proof}

\begin{problem}
Implementa\c{t}i \^{\i}n MATLAB o metod\u{a} hibrid\u{a}
secant\u{a}+\^{\i}njum\u{a}t\u{a}\c{t}ire pentru rezolvarea ecua\c{t}iei
$f(x)=0$. Testa\c{t}i pentru $f(x)=1-\frac{2}{x^{2}-2x+2}$. C\^{a}te
itera\c{t}ii sunt necesare? Compara\c{t}i cu metoda secantei \c{s}i a \^{\i}njum\u{a}t\u{a}\c{t}irii.
\end{problem}

\bigskip
\begin{verbatim}
function [z,ni]=secantsafe(f,a,b,tol)
% SECANTSAFE - safe secant method = secant + bisection
% call z=secantsafe(f,a,b,tol)

% The method uses three points a, b, and c. The points a and b are the next 
% points xk and xk?1 in the secant method approximation. 
% The points b and c form a sign change interval (proper bracket) for the root. 
% The idea behind the method is that if the secant method produces an
% undesirable approximation, we take the midpoint of the sign change interval 
% (next bisection iterate) as our next approximation.

% Let fa = f(a), fb = f(b) and fc = f(c) which must satisfy
% Conditions:
% 1. fa, fb, fc ~= 0,
% 2. sign(fb) ~= sign(fc) (sign change interval)
% 3. |fb| <= |fc|.

fa=f(a); fb=f(b);
if fa==0
    z=a; return;
end
if fb==0
    z=b; return;
end
if sign(fa)==sign(fb)
    error('f(a) and f(b) must have different sign')
end
c=a; fc=fa;
ni=0;
while 1 %forever
    ni=ni+1;
    if abs(fc) < abs(fb) %swap b and c
        t = c; c = b; b = t;
        t = fc; fc = fb; fb = t;
        % In this case, a and b may no longer
        % be a pair of secant iterates, and we must set a = c.
        a = c; fa = fc;
    end
    if abs(b-c) <= tol, break; end %success
    dm = (c-b)/2;
    df = (fa-fb);
    if df == 0
        ds = dm;
    else
        ds = -fb*(a-b)/df;
    end
    if (sign(ds)~=sign(dm) || abs(ds) > abs(dm))%bisection or secant
        dd = dm;
    else
        dd = ds;
    end
    if abs(dd) < tol
        dd = 0.5*sign(dm)*tol;
    end
    % New iterate b+dd
    d = b + dd;
    fd = f(d);
    if fd == 0
        b = d; c = d; fb = fd; fc = fd;
        break;
    end
    a = b; b = d;
    fa = fb; fb = fd;
    if sign(fb) == sign(fc)
        c = a; fc = fa;
    end
end
z=(b+c)/2;
\end{verbatim}

Test

\begin{verbatim}
f=@(x) 1-2./(x.^2-2*x+2);
%[-10,1]
[z1,ni1]=secantsafe(f,-10,1,1e-8)
[z2,ni2]=Bisection(f,-10,1,1e-8)
[z3,ni3]=secant(f,-10,1,1e-8)
\end{verbatim}

Rezultate

\begin{verbatim}
z1 =
-1.9747e-09
ni1 =
11
z2 =
-1.6298e-09
ni2 =
31
Error using secant (line 28)
numarul maxim de iteratii depasit
Error in testsecantsafe2 (line 5)
[z3,ni3]=secant(f,-10,1,1e-8) 
\end{verbatim}

\end{document}