\documentclass[a4paper]{article}%
\usepackage{amsmath}
\usepackage{amsfonts}
\usepackage{amssymb}
\usepackage{graphicx}%
\setcounter{MaxMatrixCols}{30}
%TCIDATA{OutputFilter=latex2.dll}
%TCIDATA{Version=5.50.0.2953}
%TCIDATA{CSTFile=40 LaTeX article.cst}
%TCIDATA{Created=Friday, June 09, 2017 22:43:27}
%TCIDATA{LastRevised=Sunday, June 11, 2017 08:53:53}
%TCIDATA{<META NAME="GraphicsSave" CONTENT="32">}
%TCIDATA{<META NAME="SaveForMode" CONTENT="1">}
%TCIDATA{BibliographyScheme=Manual}
%TCIDATA{<META NAME="DocumentShell" CONTENT="Standard LaTeX\Blank - Standard LaTeX Article">}
%BeginMSIPreambleData
\providecommand{\U}[1]{\protect\rule{.1in}{.1in}}
%EndMSIPreambleData
\newtheorem{theorem}{Theorem}
\newtheorem{acknowledgement}[theorem]{Acknowledgement}
\newtheorem{algorithm}[theorem]{Algorithm}
\newtheorem{axiom}[theorem]{Axiom}
\newtheorem{case}[theorem]{Case}
\newtheorem{claim}[theorem]{Claim}
\newtheorem{conclusion}[theorem]{Conclusion}
\newtheorem{condition}[theorem]{Condition}
\newtheorem{conjecture}[theorem]{Conjecture}
\newtheorem{corollary}[theorem]{Corollary}
\newtheorem{criterion}[theorem]{Criterion}
\newtheorem{definition}[theorem]{Definition}
\newtheorem{example}[theorem]{Example}
\newtheorem{exercise}[theorem]{Exercise}
\newtheorem{lemma}[theorem]{Lemma}
\newtheorem{notation}[theorem]{Notation}
\newtheorem{problem}[theorem]{Problema}
\newtheorem{proposition}[theorem]{Proposition}
\newtheorem{remark}[theorem]{Remark}
\newtheorem{solution}[theorem]{Solution}
\newtheorem{summary}[theorem]{Summary}
\newenvironment{proof}[1][Proof]{\noindent\textbf{#1.} }{\ \rule{0.5em}{0.5em}}
\begin{document}
\section*{Subiectul 9}

\begin{problem}
\label{pb4.35}

\begin{enumerate}
\item[(a)] Fie $\mathrm{d\,}\lambda$ o m\u{a}sur\u{a} simetric\u{a} pe
$[-a,a]$, $0<a\leq\infty$ \c{s}i
\[
\pi_{2k}(t;\mathrm{d\,}\lambda)=\pi_{k}^{+}(t^{2}).
\]
Ar\u{a}ta\c{t}i c\u{a} $\{\pi_{k}^{+}\}$ sunt polinoame ortogonale monice pe
$[0,a^{2}]$\ \^{\i}n raport cu m\u{a}sura $\mathrm{d\,}\lambda^{+}%
(t)=t^{-1/2}w(t^{1/2})\mathrm{d\,}t$. (1p)

\item[(b)] Aplica\c{t}i acest rezultat la calculul polinoamelor ortogonale pe
$[0,1]$ \^{\i}n raport cu ponderea $w(t)=\frac{1}{\sqrt{t}}$. (1p)

\item[(c)] Genera\c{t}i o formul\u{a} de cuadratur\u{a} de tip Gauss cu
dou\u{a} noduri pentru aceast\u{a} pondere. (4p)
\end{enumerate}
\end{problem}

\begin{problem}
\label{Gautschip4.27}Consider\u{a}m ecua\c{t}iile echivalente%
\[
(A)~x\ln x-1=0;\quad(B)~\ln x-\frac{1}{x}=0.
\]


\begin{enumerate}
\item[(a)] Ar\u{a}ta\c{t}i c\u{a} au exact o r\u{a}d\u{a}cin\u{a} pozitiv\u{a}
\c{s}i determina\c{t}i un interval care o con\c{t}ine. (1p)

\item[(b)] At\^{a}t pentru (A) c\^{a}t \c{s}i pentru (B), determina\c{t}i cel
mai mare interval pe care metoda lui Newton converge. (\emph{Indica\c{t}ie}:
studia\c{t}i convexitatea celor dou\u{a} func\c{t}ii care apar \^{\i}n
ecua\c{t}ii.) (1p)

\item[(c)] Care din cele dou\u{a} itera\c{t}ii converge asimptotic mai repede? (1p)
\end{enumerate}
\end{problem}

\newpage

\section*{Subiectul 10}

\begin{problem}
\label{pb4.35b}

\begin{enumerate}
\item[(a)] Fie $\mathrm{d\,}\lambda$ o m\u{a}sur\u{a} simetric\u{a} pe
$[-a,a]$, $0<a\leq\infty$ \c{s}i
\[
\pi_{2k+1}(t;\mathrm{d\,}\lambda)=t\pi_{k}^{-}(t^{2}).
\]
Ar\u{a}ta\c{t}i c\u{a} $\{\pi_{k}^{-}\}$ sunt polinoame ortogonale monice pe
$[0,a^{2}]$\ \^{\i}n raport cu m\u{a}sura $\mathrm{d\,}\lambda^{-}%
(t)=t^{+1/2}w(t^{1/2})\mathrm{d\,}t$. (1p)

\item[(b)] Aplica\c{t}i acest rezultat la calculul polinoamelor ortogonale pe
$[0,1]$ \^{\i}n raport cu ponderea $w(t)=\sqrt{t}$. (1p)

\item[(c)] Genera\c{t}i o formul\u{a} de cuadratur\u{a} de tip Gauss cu
dou\u{a} noduri pentru aceast\u{a} pondere. (4p)
\end{enumerate}
\end{problem}

\begin{problem}
\label{Gautschip4.30}Ecua\c{t}ia
\[
\cos x\cosh x-1=0
\]
are exact dou\u{a} r\u{a}d\u{a}cini $\alpha_{n}$ $<\beta_{n}$ \^{\i}n fiecare
interval $\left[  -\frac{\pi}{2}+2n\pi,\frac{\pi}{2}+2n\pi\right]  $,
$n=1,2,3,\dots$ (1p) Ar\u{a}ta\c{t}i c\u{a} metoda lui Newton aplicat\u{a}
aceste ecua\c{t}ii converge c\u{a}tre $\alpha_{n}$ c\^{a}nd se ia valoarea de
pornire $x_{0}=-\frac{\pi}{2}+2n\pi$ (1p) \c{s}i c\u{a}tre $\beta_{n}$
c\^{a}nd se ia valoarea de pornire $x_{0}=\frac{\pi}{2}+2n\pi$. (1p)
\end{problem}


\end{document}