
% This LaTeX was auto-generated from an M-file by MATLAB.
% To make changes, update the M-file and republish this document.

\documentclass{article}
\usepackage{graphicx}
\usepackage{color}

\sloppy
\definecolor{lightgray}{gray}{0.5}
\setlength{\parindent}{0pt}

\begin{document}

    
    
\section*{Derivare numerica}


\subsection*{Contents}

\begin{itemize}
\setlength{\itemsep}{-1ex}
   \item Deducerea aproximarii
   \item Exemplu
   \item Precizia maxima
   \item Sursa neplacerii
   \item Conditionarea absoluta
   \item Conditionarea relativa
\end{itemize}


\subsection*{Deducerea aproximarii}

\begin{par}
Utilizand formula lui Taylor
\end{par} \vspace{1em}
\begin{par}
$$f(x+h) = f(x) +hf'(x) + \frac{h^2}{2}f''(\xi), \xi \in [x,x+h]$$
\end{par} \vspace{1em}
\begin{par}
se obtine
\end{par} \vspace{1em}
\begin{par}
$$f'(x) = \frac{f(x+h)-f(x)}{h}-\frac{h}{2}f''(\xi)$$
\end{par} \vspace{1em}
\begin{par}
Termenul $-\frac{h}{2}f''(\xi)$ este \textbf{eroarea de trunchiere} sau \textbf{eroarea de discretizare} la aproximarea lui $f'(x)$ prin $\frac{f(x+h)-f(x)}{h}$. Eroarea este $O(h)$ si spunem ca precizia este de ordinul I. La derivarea numerica vom presupune ca $x+h$ si $x$ se reprezinta exact, iar erorile se comit doar la evaluarea lui $f(x+h)$ si $f(x)$. Ignorand erorile de rotunjire la scadere si impartire, se calculeaza
\end{par} \vspace{1em}
\begin{par}
$$\frac{f(x+h)(1+\delta_{1})-f(x)(1+\delta_2)}{h}=\frac{f(x+h)-f(x)}{h}+ \frac{\delta_1f(x+h)-\delta_2f(x)}{h}$$
\end{par} \vspace{1em}
\begin{par}
Deoarece $|\delta_1|<eps$ si $|\delta_2|<eps$, eroare de rotunjire este mai mica sau egala cu $2eps|f(x)|/h$, pentru $h$ mic. De notat ca eroarea de trunchiere este proportionala cu $h$, iar eroarea de rotunjire este proportionala cu $1/h$. Micsorarea lui $h$  micsoreaza eroare de trunchiere, dar creste eroarea de rotunjire.
\end{par} \vspace{1em}


\subsection*{Exemplu}

\begin{par}
Luam $f(x)=\sin x$ si $x=\pi/4$. Atunci $f'(x)=\cos x$ si $f''(x)=-\sin x$, deci eroarea de trunchiere este de aproximativ $\sqrt{2}h/4$, iar eroarea de rotunjire este de aproximativ $\sqrt{2} eps/h$
\end{par} \vspace{1em}
\begin{verbatim}
x = pi/4;
h = 10.^(-(1:16))';
d = (sin(x+h)-sin(x))./h;
[d, sqrt(2)/2*ones(size(d)), abs(d-cos(x))]
\end{verbatim}

        \color{lightgray} \begin{verbatim}
ans =

    0.6706    0.7071    0.0365
    0.7036    0.7071    0.0035
    0.7068    0.7071    0.0004
    0.7071    0.7071    0.0000
    0.7071    0.7071    0.0000
    0.7071    0.7071    0.0000
    0.7071    0.7071    0.0000
    0.7071    0.7071    0.0000
    0.7071    0.7071    0.0000
    0.7071    0.7071    0.0000
    0.7071    0.7071    0.0000
    0.7071    0.7071    0.0000
    0.7083    0.7071    0.0012
    0.7105    0.7071    0.0034
    0.7772    0.7071    0.0700
    1.1102    0.7071    0.4031

\end{verbatim} \color{black}
    

\subsection*{Precizia maxima}

\begin{par}
Precizia maxima se obtine daca cele doua erori sunt aproximativ egale
\end{par} \vspace{1em}
\begin{par}
$$\frac{\sqrt{2}h}{4} = \frac{\sqrt{2}eps}{h} \Rightarrow h=2\sqrt{eps}$$
\end{par} \vspace{1em}
\begin{par}
Eroarea este de ordinul $\sqrt{eps}$
\end{par} \vspace{1em}
\begin{verbatim}
ho = 2*sqrt(eps);
do = (sin(x+ho)-sin(x))./ho;
[ho, do]
\end{verbatim}

        \color{lightgray} \begin{verbatim}
ans =

    0.0000    0.7071

\end{verbatim} \color{black}
    

\subsection*{Sursa neplacerii}

\begin{par}
Sursa neplacerii este \textit{algoritmul} nu problema determinarii
\end{par} \vspace{1em}
\begin{par}
$$\frac{d}{dx}\left. \sin{x}\right|_{x=\pi/4} =\left. \cos x \right|_{x=\pi/4}=\frac{\sqrt{2}}{2}$$
\end{par} \vspace{1em}
\begin{par}
care este bine conditionata
\end{par} \vspace{1em}


\subsection*{Conditionarea absoluta}

\begin{par}
$$\kappa(x)=\left| -\sin x\right|_{x=\pi/4}=\frac{\sqrt{2}}{2}$$
\end{par} \vspace{1em}


\subsection*{Conditionarea relativa}

\begin{par}
$$cond(f)(x)=\left | \frac{x \sin x}{\cos x}\right|_{x=\pi/4}=\frac{\pi}{4}.$$
\end{par} \vspace{1em}



\end{document}
    
